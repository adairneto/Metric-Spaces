\chapter{The Topology of Metric Spaces}

\section{Open and Closed Sets}

In this section, we introduce some topological concepts.

\begin{definition}[Topology]
    A \textbf{topology} on a set $X$ is a family $\tau$ of subsets of $X$ with the following properties:
    \begin{itemize}
        \item $\emptyset, X \in \tau$;
        \item $A, B \in \tau \implies A \cap B \in \tau$;
        \item $\mathcal{U} \subseteq \tau \implies \bigcup \mathcal{U} \in \tau$.
    \end{itemize}

    The pair $(X, \tau)$ is called a \textbf{topological space} and its members are called the \textbf{open sets} of $X$ with respect to $\tau$.
\end{definition}

To put it another way, a topological space is closed under finite intersections and arbitrary unions.

\begin{definition}[Open and Closed Sets]
    Let $(M, \mathrm{d})$ be a metric space. A subset $A \subset M$ is said to be \textbf{open} if, for all $p \in A$, there exists $\varepsilon > 0$ such that $B(p, \varepsilon) \subset A$.

    A set $F \subset M$ is said to be \textbf{closed} if $F^c = M \setminus F$ is open.
\end{definition}

Intuitively, a set is open if all of its points fit loosely inside it.

Remark that if $A \neq \emptyset$ is an open set, then $A$ is a union of open balls. And if $A$ is a union of open balls, then $A$ is an open set.

The next proposition asserts that the collection of open sets of a metric space is a topology over $M$, called the \textbf{topology induced by a metric}.

\begin{proposition}\label{prop:opens-topology}
    Let $\mathcal{A}$ be the collection of all open sets of a metric space $(M, \mathrm{d})$. Then
    \begin{enumerate}
        \item $\emptyset, M \in \mathcal{A}$;
        \item $X, Y \in \mathcal{A} \implies X \cap Y \in \mathcal{A}$;
        \item If $(X_i)$ is a family of open sets of $M$, i.e., each $X_i \in \mathcal{A}$, then $\bigcup X_i \in \mathcal{A}$.
    \end{enumerate}
\end{proposition}

\begin{proof}
    \begin{enumerate}
        \item $\emptyset, M \in \mathcal{A}$ trivially.
        \item Let $X, Y \in \mathcal{A}$ and $p \in X \cap Y$. 
        
        Since $X$ is open, there exists $\varepsilon_1 > 0$ such that $B(p, \varepsilon_1) \subset X$.

        Since $Y$ is open, there exists $\varepsilon_2 > 0$ such that $B(p, \varepsilon_2) \subset Y$.

        Take $\varepsilon = \min \{ \varepsilon_1, \varepsilon_2 \}$. Given that $B(p, \varepsilon) \subset B(p, \varepsilon_1) \cap B(p, \varepsilon_2)$, we know that $B(p, \varepsilon) \subset X \cap Y$. Hence, $X \cap Y$ is an open set of $\mathcal{A}$.

        \item Let $\{ X_i \}_{i \in I} \subset \mathcal{A}$ and $X = \bigcup_{i \in I} X_i$. Taking $p \in X$, there exists $i \in I$ such that $p \in X_i$. Therefore, there is $\varepsilon > 0$ such that $B(p, \varepsilon) \subset X_1 \subset X$. Thus, $X$ is an open set of $\mathcal{A}$.
    \end{enumerate}
\end{proof}

Equivalent metrics determine the same topological structure.

\begin{proposition}
    Let $\mathrm{d}$ and $\mathrm{d}'$ be equivalent metrics over $M$. If $\mathcal{A}$ is a collection of open sets of $(M, \mathrm{d})$ and $\mathcal{A}'$ is a collection of open sets of $(M, \mathrm{d}')$, then $\mathcal{A} = \mathcal{A}'$.
\end{proposition}

\begin{proof}
    Let $\mathcal{A} \subset \mathcal{A}'$ and $A \in \mathcal{A}$. Given $p \in A$, there exists $\varepsilon > 0$ such that $B_{\mathrm{d}}(p, \varepsilon) \subset A$.

    Since $d \sim d'$, there exists $\delta > 0$ such that
    \[
        B_{\mathrm{d}'}(p, \delta) \subset B_{\mathrm{d}}(p, \varepsilon)
    \]

    Hence, $B_{\mathrm{d}'}(p, \delta) \subset A$ and, therefore, $A \in \mathcal{A}'$.
\end{proof}

\begin{definition}[Interior Point]
    Let $(M, \mathrm{d})$ be a metric space and $A \subset M$. A point $p \in A$ is called an \textbf{interior point} of $A$ if there exists $\varepsilon > 0$ such that $B(p, \varepsilon) \subset A$. The set of interior points of $A$ is called the \textbf{interior} of $A$.
\end{definition}

With this definition, an equivalent notion of open sets follows: a set $A$ is open iff. every point of $A$ is interior.

\begin{proposition}
    Let $\mathcal{F}$ be the collection of all closed sets of a metric space $(M, \mathrm{d})$. Then
    \begin{enumerate}
        \item $\emptyset, M \in \mathcal{F}$;
        \item $X, Y \in \mathcal{F} \implies X \cup Y \in \mathcal{F}$;
        \item If $(X_i)$ is a family of open sets of $M$, i.e., each $X_i \in \mathcal{F}$, then $\bigcap X_i \in \mathcal{F}$.
    \end{enumerate}
\end{proposition}

\begin{proof}
    \begin{enumerate}
        \item $\emptyset, M \in \mathcal{F}$ trivially.
        \item $X, Y \in \mathcal{F} \implies X^c, Y^c \in \mathcal{A} \implies X^c \cap Y^c \in \mathcal{A} \implies (X \cup Y)^c \in \mathcal{A} \implies X \cup Y \in \mathcal{F}$.
        \item $\{ X_i \}_{i \in I} \subset \mathcal{F} \iff \{ X_i^c \}_{i \in I} \subset \mathcal{A} \implies \bigcup_{i \in I} X_i^c \in \mathcal{A} \implies \bigcap_{i \in I} X_i \in \mathcal{F}$.
    \end{enumerate}
\end{proof}

\section{Adherence, Accumulation and Closure}

This section investigates when a point is `close' to a set.

\begin{definition}[Adherent point]
    Let $A \subset M$. A point $p \in M$ is called an \textbf{adherent point} of $A$ if, for all $\varepsilon > 0$, we have
    \[
        B(p, \varepsilon) \cap A \neq \emptyset
    \]

    The set of all adherent points of $A$ is called the \textbf{closure} of $A$ and is denoted by $\overline{A}$.
\end{definition}

The intuition behind this definition is that every neighborhood of $p$ contains at least one point of $A$.

\begin{proposition}
    The complement of the closure equals the interior of the complement, i.e.,
    \[
        (\overline{A})^c = \text{int}(A^c)
    \]
\end{proposition}

\begin{proof}
    \begin{equation*}
        \begin{aligned}
            p \in (\overline{A})^c &\iff p \notin \overline{A} \\
            &\iff \exists \varepsilon > 0 : B(p, \varepsilon) \cap A = \emptyset \\
            &\iff \exists \varepsilon > 0: B(p, \varepsilon) \subset A^c \\
            &\iff p \in \text{int}(A^c)
        \end{aligned}
    \end{equation*}
\end{proof}

\begin{corollary}
    A set is closed iff. it is equal to its closure.
\end{corollary}

\begin{proof}
    We know that $F$ is closed iff. $F^c$ is open, which means that $\text{int}(F^c) = F^c$. By the preceeding proposition, $(\overline{F})^c = F^c$, i.e., $\overline{F} = F$.
\end{proof}

\begin{proposition}
    Let $(M, \mathrm{d})$ be a metric space. If $p \in M$ and $A \subset M$, then $\mathrm{d}(p, A) = 0$ iff. $p \in \overline{A}$.
\end{proposition}

\begin{proof}
    $(\Rightarrow)$ Apply the definition of distance from a point to a set and equal it to zero. Show that there is $a \in A$ such that $a \in B(p, \varepsilon)$.

    $(\Leftarrow)$ Suppose that the distance equals $\varepsilon > 0$. By hypothesis, there exists $a \in A$ such that $\mathrm{d}(a, p) < \varepsilon$ and derive a contradiction.
\end{proof}

\begin{proposition}
    For all non-empty subsets $A$ of $M$, we have $\mathrm{d}(A) = \mathrm{d}(\overline{A})$.
\end{proposition}

\begin{proof}
    Since $A \subseteq \overline{A}$, it follows that $\mathrm{d}(A) \leq \mathrm{d}(\overline{A})$. Take $\varepsilon > 0$ and $x, y \in \overline{A}$.

    Thus, there exists $a, b \in A$ such that 
    \[
        a \in B(x, \varepsilon/2) \text{ and } b \in B(y, \varepsilon/2)
    \]
    I.e.,
    \[
        \mathrm{d}(a,x) < \frac{\varepsilon}{2} \text{ and } \mathrm{d}(y,b) < \frac{\varepsilon}{2}
    \]

    By the triangular inequality,
    \begin{equation*}
        \begin{aligned}
            \mathrm{d}(x,y) &\leq \mathrm{d}(x,a) + \mathrm{d}(a,y) \\
            &\leq \mathrm{d}(x,a) + \mathrm{d}(a,b) + \mathrm{d}(b,y) \\
            &< \frac{\varepsilon}{2} + \frac{\varepsilon}{2} + \mathrm{d}(a,b) \\
            &< \mathrm{d}(A) + \varepsilon
        \end{aligned}
    \end{equation*}

    Therefore, $\mathrm{d}(\overline{A}) < \mathrm{d}(A) + \varepsilon$, which means that $\mathrm{d}(\overline{A}) \leq \mathrm{d}(A)$. Hence, $\mathrm{d}(A) = \mathrm{d}(\overline{A})$.
\end{proof}

A closure is a set plus its `boundary'. Naturally, the boundary of a set $A$ is formed by the points $p \in M$ such that every open ball centered on $p$ contains at least one point of $A$ and one point of the complement $M \setminus A$.

\begin{definition}[Boundary]
    Let $(M, \mathrm{d})$ be a metric space and $A \subset M$. The \textbf{boundary} (or \textbf{frontier}) of $A$ is the set
    \[
        \partial A = \{ p \in M : B(p, \varepsilon) \cap A \neq \emptyset \text{ and } B(p, \varepsilon) \cap (M \setminus A) \neq \emptyset, \, \forall \varepsilon > 0 \}
    \]
\end{definition}

\begin{proposition}
    Let $(M, \mathrm{d})$ be a metric space and $A \subset M$. Then $\overline{A} = A \cup \partial A$.
\end{proposition}

\begin{proof}
    Let $a \in \overline{A}$ and $a \notin A$. Using the definition of adherence point, we know that $a \in \partial A$.

    Let $a \in A \cup \partial A$. If $a \in A$, then $a \in \overline{A}$. If $a \notin A$, then $a \in \partial A$, hence $a \in \overline{A}$.
\end{proof}

\begin{proposition}
    Let $(M, \mathrm{d})$ be a metric space. If $A \subset M$ and $p \in \overline{A}$, then there exists a sequence $(x_n)$ of points in $A$ such that $\lim x_n = p$.
\end{proposition}

\begin{proof}
    For each $n \in \mathbb{N}$, let $x_n \in B(p, 1/n) \cap A$. I.e., $\mathrm{d}(p, x_n) < 1/n$. Given $\varepsilon > 0$, let $n_0 \in \mathbb{N}$ such that $1/n_0 < \varepsilon$. Then,
    \[
        n \geq n_0 \implies \mathrm{d}(p, x_n) < \frac{1}{n} \leq \frac{1}{n_0} < \varepsilon
    \]
\end{proof}

\begin{definition}[Dense]
    Given a metric space $(M, \mathrm{d})$, a subset $A \subset M$ is said to be \textbf{dense} in $M$ if $\overline{A} = M$.
\end{definition}

The idea here is that for all points $p \in M$ there exists another point $a \in A$ arbitrarily close to $p$. E.g. $\mathbb{Q}$ is dense in $\mathbb{R}$.

\begin{proposition}
    Let $M$ be a metric space. If $A \subset M$ is dense in $M$, then $G \cap A \neq \emptyset$ for all open set $G \neq \emptyset$ of this space.
\end{proposition}

\begin{proof}
    Let $G$ be an open set and $p \in G$. There exists $\varepsilon > 0$ such that $B(p, \varepsilon) \subset G$. Since $A$ is dense in $M$,
    \[
        A \cap B(p, \varepsilon) \neq \emptyset \implies G \cap A \supset A \cap B(p, \varepsilon) \neq \emptyset
    \]

    Hence, $G \cap A \neq \emptyset$.
\end{proof}

The following definition adds a restriction to our definition of adherence point. This will be useful when studying limits and continuity.

\begin{definition}[Accumulation point]
    Let $A \subset M$. A point $p \in M$ is called an \textbf{accumulation point} of $A$ if $p$ is an adherent point of $A \setminus \{ p \}$, i.e., for all $\varepsilon > 0$, we have that
    \[
        B(p, \varepsilon) \cap A \setminus \{ p \} \neq \emptyset
    \]

    The set of accumulation points of $A$ is called \textbf{derived set} of $A$ and is denoted by $A'$.
\end{definition}

\begin{example} \hfill
    \begin{enumerate}
        \item $A \subset B \implies A' \subset B'$ and $(a,b)' = [a,b]$.
        \item $A = [0,1] \cup \{ 2 \}$. Then $A' = [0,1]$.
        \item $A = (0,1) \cap \mathbb{Q}$. Then $A' = [0,1]$.
    \end{enumerate}
\end{example}

\begin{proposition}
    Let $p$ be an accumulation point of a set $A$. Then every ball centered at $p$ has infinitely many points of $A$.
\end{proposition}

\begin{proof}
    Suppose that there exists $\varepsilon > 0$ such that $B(p, \varepsilon) \cap A = \{ x_1, \ldots, x_n \}$, i.e., is a finite set.

    Let
    \[
        (B(p, \varepsilon) \setminus \{ p \}) \cap A = \{ y_1, \ldots, y_m \}
    \]
    where $y_i \neq y_j$ for $i \neq j$. And define 
    \[
        \delta = \min \{ \mathrm{d}(p, y_j) : 1 \leq j \leq m \} > 0
    \]

    Therefore,
    \[
        (B(p, \delta) \setminus \{ p \}) \cap A = \emptyset
    \]
    implying that $p$ is not an accumulation point of $A$, which contradicts our hypothesis.
\end{proof}

\begin{example}
    Let $M \neq \emptyset$ equipped with the zero-one metric. Let $A \subset M$ and $p \in M$. Then
    \[
        B(p, 1) = \{ p \} \text{ and } B(p, 1) \setminus \{ p \} = \emptyset
    \]

    Hence, $p$ is not an accumulation point of $A$, i.e., $A' = \emptyset$.
\end{example}

\begin{remark}
    Let $(M, \mathrm{d})$ be a metric space, $A \subset M$ and $A$ finite. Then $A' = \emptyset$.
\end{remark}

\begin{proposition}
    Let $(x_n)$ be a sequence of $(M, \mathrm{d})$ such that $A = \{ x_n : n \in \mathbb{N} \}$ has an accumulation point $p$. Then there exists a subsequence of $\{ x_n \}$ which converges to $p$.
\end{proposition}

\begin{proof}
    For each $j \in \mathbb{N}$ define 
    \[
        C_j = B(p, 1/j) \cap A
    \]
    and notice that $C_j$ is infinite. Choose $x_{n_1} \in C_1$, $x_{n_2} \in C_2$ such that $x_{n_2} \neq x_{n_1}$ and $n_2 > n_1$.

    Suppose that we already chose $x_{n_k} \in C_k$ such that $x_{n_i} \neq x_{n_l}$ for all $i \neq l$ and $n_1 < n_2 < \ldots < n_{k-1} < n_k$.

    We now choose $x_{n_{k+1}}$ such that $x_{n_{k+1}} \in C_{k+1}$. $x_{n_{k+1}} \notin \{ x_{n_1}, \ldots, x_{n_k} \}$ and $n_{k+1} > n_k$.

    By the induction principle, we construct a subsequence $(x_{n_j})_{j \in \mathbb{N}}$ of $(x_n)$.

    And we have that
    \[
        x_{n_j} \in C_j \iff \mathrm{d}(p, x_{n_j}) < \frac{1}{j} \implies x_{n_j} \longrightarrow p \text{ as } j \to \infty
    \]
\end{proof}

\begin{proposition}
    A set $F \subset M$ is closed iff. $F' \subset F$.
\end{proposition}

\begin{proof}
    $(\Rightarrow)$ Suppose, by contradiction, that $F$ is closed, but $F' \not\subset F$. Therefore, there exists $p \in F'$ such that $p \notin F$. Then $p \in F' \setminus F = F' \cap F^c$.
    
    Since $F^c$ is open, there exists $\varepsilon > 0$ such that $B(p, \varepsilon) \subset F^c$. I.e., $B(p, \varepsilon) \cap F = \emptyset$. Hence, $p \notin F'$.

    $(\Leftarrow)$ Let $p \in F^c$. Then $p \notin F$ and $p \notin F'$. Then there exists $\varepsilon > 0$ such that 
    \[
        (B(p, \varepsilon) \setminus \{ p \}) \cap F = \emptyset
    \]

    Since $p \notin F$, 
    \[
        B(p, \varepsilon) \cap F = \emptyset \implies B(p, \varepsilon) \subset F^c
    \]
    Hence, $F^c$ is open, i.e., $F$ is closed.
\end{proof}